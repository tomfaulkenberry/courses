\documentclass[12pt]{article}%
\newcommand{\thisdocument}{}%
\newcommand{\thiscourse}{PSYC 3330: Elem Stats for Behav Sci}
\newcommand{\thissemester}{Week 10}
\usepackage{amsmath}%
\usepackage{amsfonts}%
\usepackage{amssymb}%
\usepackage{amsthm}%

%\textwidth = 6.5 in
%\textheight = 8 in
%\oddsidemargin = 0.0 in
%\evensidemargin = 0.0 in
%\topmargin = 0.0 in
%\headheight = 0.0 in
%\headsep = 0.0 in
\parskip = 0.1in
\parindent = 0.0in

\usepackage[left=1in,right=1in, top=1in, bottom=0.5in]{geometry}

\usepackage{fancyhdr,ifthen}
\pagestyle{fancy}
\lfoot{}  % no footers (in pagestyle fancy)
\cfoot{}
% running left heading
\lhead{\bfseries\Large\em \noindent{}\hspace{-.2em}\thiscourse{}\\[1mm]}
% running right heading
%\newcommand{\spc}{1.31em}
\newcommand{\spc}{1em}

\rhead{\em \sf Tarleton State University \hfill \sf \thisdocument{}}

\setlength{\headheight}{3ex}
\newcommand{\mainhead}[1]{\begin{center}{\Large \bf #1}\end{center}}
\newcommand{\head}[1]{\vspace{1.5ex}\par\noindent{\large \bf #1}\par\noindent}
\newcommand{\subhead}[1]{\vspace{2ex}\par\noindent{\sl #1}\vspace{1ex}\par\noindent{}}
\newcommand{\ptitle}{\sl}

\newcommand{\hra}{\hookrightarrow}


%%%% Theoremstyles
\theoremstyle{plain}
\newtheorem{theorem}{Theorem}[section]
\newtheorem{proposition}[theorem]{Proposition}
\newtheorem{corollary}[theorem]{Corollary}
\newtheorem{claim}[theorem]{Claim}
\newtheorem{lemma}[theorem]{Lemma}
\newtheorem{conjecture}[theorem]{Conjecture}

\theoremstyle{definition}
\newtheorem{definition}[theorem]{Definition}
\newtheorem{algorithm}[theorem]{Algorithm}
\newtheorem{question}[theorem]{Question}
\newtheorem{problem}[theorem]{Problem}
\newtheorem{goal}[theorem]{Goal}

\theoremstyle{remark}
\newtheorem{remark}[theorem]{Remark}
\newtheorem{remarks}[theorem]{Remarks}
\newtheorem{example}[theorem]{Example}
\newtheorem{exercise}[theorem]{Exercise}


\DeclareMathOperator{\SO}{SO}%
\DeclareMathOperator{\Sp}{Sp}%
\DeclareMathOperator{\SL}{SL}%
\DeclareMathOperator{\End}{End}%
\DeclareMathOperator{\Tr}{Tr}%
\DeclareMathOperator{\Res}{Res}%
\DeclareMathOperator{\res}{res}%
\DeclareMathOperator{\BSD}{BSD}%
\DeclareMathOperator{\Gal}{Gal}%
\DeclareMathOperator{\GL}{GL}%
\DeclareMathOperator{\Aut}{Aut}%
\DeclareMathOperator{\Reg}{Reg}%
\DeclareMathOperator{\Vis}{Vis}%
\DeclareMathOperator{\Ker}{Ker}%
\DeclareMathOperator{\Coker}{Coker}%
\DeclareMathOperator{\Sel}{Sel}%
\DeclareMathOperator{\ord}{ord}%
\DeclareMathOperator{\new}{new}%
\DeclareMathOperator{\an}{an}%

\newcommand{\abcd}[4]{\left(
        \begin{smallmatrix}#1&#2\\#3&#4\end{smallmatrix}\right)}

\usepackage{graphicx}
\setlength{\parskip}{1mm}

\begin{document}
\mbox{}
\large
\begin{enumerate}

 \item Dr. Mnemonic develops a new treatment for patients with a memory disorder. He isn't certain what impact, if any, it will have. To test it, he randomly assigns 8participants to participants to one of two samples.  He gives one sample the new treatment, but not the other.  Following the treatment period, he gives both groups a memory test.  Using the data below, decide whether the memory treatment had a significant effect. Use a two-tailed test with $\alpha=0.05$.

\begin{table}[h!]
\begin{center}
\begin{tabular}{rr}
Experimental group & Control Group\\
\hline
45 & 43\\
55 & 49\\
40 & 35\\
60 & 51\\
\hline
$\overline{x}_{A}=50$ & $\overline{x}_B=44.5$\\
$SS_A=250$ & $SS_B=155$\\
\end{tabular}
\end{center}
\end{table}

\newpage

\item When people learn a new task, their performance usually improves when they are tested the next day, but only if they get at least 6 hours of sleep (Stickgold et al., 2000).  The following data demonstrate this phenomenon.  S's learned a visual discrimination task on one day and then were tested on the task the following day.  Half of the participants were allowed to have at least 6 hours of sleep and the other half were kept awake all night.  Is there a significant difference between the two conditions?  Use a two-tailed test with $\alpha$=0.05.

\begin{table}[h!]
	\begin{center}
	\begin{tabular}{cp{1cm}c}
		\multicolumn{3}{c}{Performance scores}\\
		6 hours sleep & & No sleep\\
		\hline
		$n=8$ & & $n=8$\\
		$M=72$ & & $M=61$\\
		$SS=440$ & & $SS=456$\\
		\hline
		
	\end{tabular}
	\end{center}
	
\end{table}

\newpage


\item In a classic study of problem solving, Duncker (1945) asked participants to mount a candle on a wall in an upright position so that it would burn normally.  One group was given a candle, a book of matches, and a box of tacks.  A second group was given the same items, except that the tacks and the box were presented separately as two distinct items.  The solution to this problem involves using the tacks to mount the box on the wall, creating a shelf for the candle.  Duncker reasoned that the first group of participants would have trouble seeing a ``new'' function for the box (a shelf) because it was already serving a function (holding tacks).  For each participant, the amount of time to solve the problem was recorded.  Data similar to Duncker's are as follows:

\begin{table}[h!]
	\begin{center}
	\begin{tabular}{cp{1cm}c}
		\multicolumn{3}{c}{Time to solve problem (in sec.)}\\
		Box of tacks & & Tacks and box separate\\
		\hline
		128 & & 42\\
		160 & & 24\\
		113 & & 68\\
		101 & & 35\\
		94 & & 47\\
		\hline
		
	\end{tabular}
	\end{center}

\end{table}

Do the data indicate a significant difference between the two conditions?  Test at the 0.01 level of significance.

\newpage
.
\end{enumerate}

\end{document}
