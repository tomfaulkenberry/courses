\documentclass[10pt]{article}
\usepackage[left=1in,right=1in,top=1in,bottom=1in]{geometry}
\usepackage{amsmath}
\usepackage{fancyhdr}
\lhead{PSYC 5316}
\chead{Homework 8}
\cfoot{}
\rhead{Fall 2017}

\pagestyle{fancy}

\begin{document}

\begin{enumerate}
  
\item Recall the globe tossing model from the lecture.  Compute and plot the grid approximate posterior distribution for each of the following sets of observations.  In each case, assume a uniform prior for $p$, and use a grid of $n=30$ points.
  \begin{enumerate}
  \item W, W, W
  \item W, W, W, L
  \item L, W, W, L, W, W, W
  \end{enumerate}

\item Now assume a prior for $p$ that is equal to 0 when $p<0.5$ and is a positive constant when $p>0.5$.  Again, compute and plot the grid approximate posterior distribution for each of the sets of observations in the problem above.

\item Compute a grid approximate posterior for the globe tossing model using $n=1000$ points.  Use the same flat prior as before.  Then, draw 10,000 samples from the posterior and answer the following questions:
  \begin{enumerate}
  \item How much posterior probability lies below $p=0.2$?
  \item How much posterior probability lies below $p=0.8$?
  \item How much posterior probability lies between $p=0.2$ and $p=0.8$?
  \item 20\% of the posterior probability lies below which value of $p$?
    \item 20\% of the posterior probability lies above which value of $p$? 
    \end{enumerate}

 \item Suppose the globe tossing data had turned out to be 8 waters in 15 tosses.  Construct a posterior distribution, using grid approximation with $n=1000$ points.  Use the same flat prior as before.  Then, draw 10,000 samples from the posterior distribution, and compute a 90\% credible interval for $p$.  Interpret what this interval means.
\end{enumerate}  
\end{document}