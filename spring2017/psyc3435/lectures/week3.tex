% Created 2017-01-29 Sun 07:58
\documentclass[11pt]{article}
\usepackage[utf8]{inputenc}
\usepackage[T1]{fontenc}
\usepackage{fixltx2e}
\usepackage{graphicx}
\usepackage{longtable}
\usepackage{float}
\usepackage{wrapfig}
\usepackage{rotating}
\usepackage[normalem]{ulem}
\usepackage{amsmath}
\usepackage{textcomp}
\usepackage{marvosym}
\usepackage{wasysym}
\usepackage{amssymb}
\usepackage{hyperref}
\tolerance=1000
\date{Jan 30-Feb 3, 2017}
\title{Week 3 lecture notes - PSYC 3435}
\hypersetup{
  pdfkeywords={},
  pdfsubject={},
  pdfcreator={Emacs 25.1.1 (Org mode 8.2.10)}}
\begin{document}

\maketitle

\section*{Building a good research question}
\label{sec-1}
\begin{itemize}
\item Suppose we want to claim that people perform best with a good night of sleep
\begin{enumerate}
\item must FOCUS the idea
\begin{itemize}
\item break more general idea into smaller, more specific ideas
\begin{itemize}
\item what do we mean by "performance" -- academic? physical?
\item what do we mean by "good night of sleep" -- 8 hours? uninterrupted? 3 hours of REM?
\end{itemize}
\item develop underlying theoretical model
\begin{itemize}
\item since consolidation of memories happens during REM sleep, we can predict that getting more REM sleep should lead to better recall
\end{itemize}
\end{itemize}
\item must EVALUATE the idea
\begin{itemize}
\item ROT test -- is it:
\begin{itemize}
\item replicable?
\item observable?
\item testable?
\end{itemize}
\end{itemize}
\item TEST the idea
\begin{itemize}
\item what are the variables of interest?
\item what is the hypothesized relationship between these variables?
\item how should we test it?
\end{itemize}
\end{enumerate}
\end{itemize}

\section*{Some definitions}
\label{sec-2}
\begin{itemize}
\item Operational definition - a definition of an abstract concept that is formulated in terms of how the concept is being measured
\begin{itemize}
\item Example: define "memory ability" as score on a memory test
\end{itemize}
\item External validity - the degree to which results of a study apply (generalize) to individuals/behaviors outside context of the study
\item Internal validity - the degree to which a study provides \textbf{causal} information about behavior
\item Reliability - the degree to which the results of a study can be replicated under similar conditions
\end{itemize}

\section*{General research methods}
\label{sec-3}
\begin{enumerate}
\item Naturalistic observation: observation and description within a natural setting
\begin{itemize}
\item high external validity
\item hard to do well
\begin{itemize}
\item reactivity effects
\item takes a long time
\item need multiple observers
\end{itemize}
\end{itemize}
\item Survey methods -- questionnaires/interviews that ask people to provide information about themselves
\begin{itemize}
\item widely used
\item best way to collect particular kinds of data (descriptive data, preferences, etc.)
\item large amounts of data very quickly
\item difficult to do correctly with high validity
\end{itemize}
\item Archival data - examine existing public/private records

\item Note: these are all called "observation without manipulation"
\begin{itemize}
\item Advantages:
\begin{enumerate}
\item can observe complex patterns of behavior
\item useful when little is known about subject of study
\item may learn something new that would never have been predicted
\end{enumerate}
\item Disadvantages:
\begin{enumerate}
\item cannot establish causality
\item threats to internal validity
\begin{itemize}
\item lots of "confounds"
\item lots of alternative explanations
\end{itemize}
\item sometimes results are not reproducible
\end{enumerate}
\end{itemize}

\item Correlational methods -- measure two (or more) variables for each individual and see if the variables co-vary (suggesting they are related)
\begin{itemize}
\item used for
\begin{itemize}
\item making predictions
\item establishing reliability and validity
\end{itemize}
\item problem: cannot make causal claims!
\end{itemize}

\item Experiments -- manipulating and controlling variables in laboratory experiments
\begin{itemize}
\item involves some type of comparison
\begin{itemize}
\item at least two groups that get compared (random assignment to groups)
\item quasi-experiment - groups NOT randomly assigned
\end{itemize}
\item types of variables
\begin{itemize}
\item Independent variable (IV) -- variable that is manipulated
\item Dependent variable (DV) -- variable that is measured
\item Control variable -- held constant for all participants in experiment (either through explicit control or randomization)
\end{itemize}
\end{itemize}
\end{enumerate}
% Emacs 25.1.1 (Org mode 8.2.10)
\end{document}