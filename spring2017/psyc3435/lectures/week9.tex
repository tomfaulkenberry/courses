% Created 2017-03-19 Sun 19:07
\documentclass[11pt]{article}
\usepackage[utf8]{inputenc}
\usepackage[T1]{fontenc}
\usepackage{fixltx2e}
\usepackage{graphicx}
\usepackage{longtable}
\usepackage{float}
\usepackage{wrapfig}
\usepackage{rotating}
\usepackage[normalem]{ulem}
\usepackage{amsmath}
\usepackage{textcomp}
\usepackage{marvosym}
\usepackage{wasysym}
\usepackage{amssymb}
\usepackage{hyperref}
\tolerance=1000
\date{Mar 20-24, 2017}
\title{Week 9 lecture notes - PSYC 3435}
\hypersetup{
  pdfkeywords={},
  pdfsubject={},
  pdfcreator={Emacs 25.1.1 (Org mode 8.2.10)}}
\begin{document}

\maketitle
This week, we'll talk about factorial designs

\section*{Motivation}
\label{sec-1}
So far, all of our experiments have had exactly \textbf{one} independent variable.  What happens when you have two or more?

\subsection*{Example}
\label{sec-1-1}
Suppose you are interested in the effects of modality (visual versus auditory) on memory.  Consider the following experiment:
\begin{itemize}
\item procedure: give participants a list of words to remember
\item IV1: presentation mode (visual, auditory)
\item IV2: test mode (visual, auditory)
\item DV: score on a recognition test ("was this word in the list?")
\end{itemize}

One way to approach this is to run two t-tests on memory scores:
\begin{itemize}
\item Test 1: compare means for visual study and auditory study
\item Test 2: compare means for visual test and auditory test
\end{itemize}

Here's some data (use JASP dataset?):
\begin{itemize}
\item Study mode:
\begin{itemize}
\item visual study: M = 65
\item auditory study: M = 65
\end{itemize}

\item Test mode:
\begin{itemize}
\item visual test: M = 65
\item auditory test: M = 65
\end{itemize}

\item what would we say about the results of our manipulations??
\begin{itemize}
\item no effect of study mode AND no effect of test mode!
\end{itemize}
\end{itemize}

However, we might be missing something:
\begin{itemize}
\item what happens when we consider various combinations of our IVs?
\end{itemize}

Data:

\begin{center}
\begin{tabular}{lrr}
 & visual pres & auditory pres\\
\hline
visual test & 80 & 50\\
auditory test & 50 & 80\\
\end{tabular}
\end{center}

\begin{itemize}
\item what happens when we plot these?
\item what story does this data tell?
\item this is a textbook example of an \emph{interaction}, and is the result of a \textbf{factorial design}
\end{itemize}

\section*{Definitions - Factorial Desins}
\label{sec-2}
\begin{itemize}
\item \emph{factors} - another name for independent variable
\begin{itemize}
\item ex: test mode
\end{itemize}
\item \emph{levels} - the values each factor can take
\begin{itemize}
\item ex: test mode has two levels: visual, auditory
\end{itemize}
\item an N x M factorial design has \emph{two} factors; the first with N levels and the second with M levels
\item Example: in our previous study, we had a 2 x 2 design.
\item Example: consider a 2 x 4 design:
\end{itemize}

\begin{center}
\begin{tabular}{lllll}
 & B1 & B2 & B3 & B4\\
\hline
A1 &  &  &  & \\
A2 &  &  &  & \\
\end{tabular}
\end{center}

The number of \emph{conditions} is calculated by multiplying the numbers of levels, so a 2x4 design has 8 conditions.
% Emacs 25.1.1 (Org mode 8.2.10)
\end{document}