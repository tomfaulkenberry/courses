% Created 2017-04-03 Mon 09:26
\documentclass[11pt]{article}
\usepackage[utf8]{inputenc}
\usepackage[T1]{fontenc}
\usepackage{fixltx2e}
\usepackage{graphicx}
\usepackage{longtable}
\usepackage{float}
\usepackage{wrapfig}
\usepackage{rotating}
\usepackage[normalem]{ulem}
\usepackage{amsmath}
\usepackage{textcomp}
\usepackage{marvosym}
\usepackage{wasysym}
\usepackage{amssymb}
\usepackage{hyperref}
\tolerance=1000
\date{April 3-7, 2017}
\title{Week 11 lecture notes - PSYC 3435}
\hypersetup{
  pdfkeywords={},
  pdfsubject={},
  pdfcreator={Emacs 25.1.1 (Org mode 8.2.10)}}
\begin{document}

\maketitle
This week, we'll talk about correlational designs

\section*{Definitions}
\label{sec-1}
\begin{itemize}
\item A \emph{correlational study} is a type of research design that examines the relationships between multiple variables
\begin{itemize}
\item Note: there is no manipulation involved, so this is a \emph{nonexperimental} design
\end{itemize}

\item Two types of questions involved:
\begin{enumerate}
\item descriptive questions: is there a relationship between different behaviors?
\item predictive questions: can one behavior be predicted from another behavior?
\end{enumerate}
\end{itemize}

To motivate our discussion, let's take a look at a dataset (albumSales.csv) that shows several variables related to album sales for 200 different metal bands

Column definitions:
\begin{itemize}
\item Adverts: advertising budget (in \$1000 units)
\item Sales: total album sales (in \$1000 units)
\item Airplay: number of radio plays per week
\item Attract: a 1-10 rating of the band's attractiveness
\end{itemize}


\section*{Testing descriptive hypotheses}
\label{sec-2}

Example (from Lab 1): is album sales related to advertising budget?
\begin{itemize}
\item the most common method for testing this relationship is a \emph{Pearson correlation} test
\item Null: the correlation between sales and advertising budget is 0 ($r=0$)
\item Alternative: the correlation between sales and advertising budget is not 0 ($r\neq 0$)
\item Results from JASP: $r = 0.578$, $p < 0.001$
\item since $p<0.05$, we reject the null, from which we conclude that the correlation between sales and budget is nonzero.
\end{itemize}

\section*{Testing predictive hypotheses}
\label{sec-3}

Predictive hypotheses are composed of a hypothesized relationship between two classes of variables:
\begin{itemize}
\item outcome variables
\item predictor variables
\end{itemize}

Typical model of predictive hypotheses is the \emph{linear regression model}: $Y=b_0+b_1X$
\begin{itemize}
\item $b_0$: intercept
\item $b_1$: slope (amount of change in Y that is due to a change in X)
\end{itemize}

Example: does advertising budget \emph{predict} album sales?
\begin{itemize}
\item look at albumSales.csv in JASP
\begin{itemize}
\item Sales = total album sales (in thousands of pounds)
\item Adverts = advertising budget (in thousands of pounds)
\end{itemize}
\item build linear model: $\text{Sales} = b_0 + b_1\cdot \text{Adverts}$
\item JASP output:
\begin{itemize}
\item $R = 0.578$: this is the correlation between sales and advertising budget
\item $R^2 = 0.335$: proportion of variation in album sales that is explained by advertising budget
\item $b_0 = 167.68$
\item $b_1 = 0.096$
\end{itemize}
\item this results in the following model:
\begin{itemize}
\item $\text{Sales} = 167.68 + 0.096\cdot\text{Adverts}$
\end{itemize}
\item Meaning?
\begin{enumerate}
\item we can predict sales from adverts: suppose advertising budget = 100,000 dollars.  This means Adverts=100.  Then \emph{predicted} sales = $167.68+0.096\cdot 100=177.28$.  Thus, we would predict total album sales of 177,280
\item we can describe the \emph{effect} of each predictor:
\begin{itemize}
\item as advertising budget increases by one unit, album sales increases by 0.096 units.
\item since unit = 1000, each \$1000 increase in budget increases album sales by $0.096\cdot 1000$ = 96 albums
\end{itemize}
\end{enumerate}
\end{itemize}
% Emacs 25.1.1 (Org mode 8.2.10)
\end{document}