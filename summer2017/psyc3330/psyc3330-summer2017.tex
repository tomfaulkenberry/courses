% Created 2017-05-19 Fri 12:32
\documentclass[10pt]{article}
\usepackage[utf8]{inputenc}
\usepackage[T1]{fontenc}
\usepackage{fixltx2e}
\usepackage{graphicx}
\usepackage{longtable}
\usepackage{float}
\usepackage{wrapfig}
\usepackage{rotating}
\usepackage[normalem]{ulem}
\usepackage{amsmath}
\usepackage{textcomp}
\usepackage{marvosym}
\usepackage{wasysym}
\usepackage{amssymb}
\usepackage{hyperref}
\tolerance=1000
\usepackage[left=1in,right=1in,bottom=1in,top=1in]{geometry}
\date{Spring 2017}
\title{PSYC 3330: Elementary Statistics for the Behavioral Sciences}
\hypersetup{
  pdfkeywords={},
  pdfsubject={},
  pdfcreator={Emacs 25.2.1 (Org mode 8.2.10)}}
\begin{document}

\maketitle

\section*{Contact info}
\label{sec-1}
\begin{itemize}
\item Professor: Thomas J. Faulkenberry, Ph.D
\item Office: Math 319
\item Office hours: MTWRF 9-11 am
\item Email: faulkenberry@tarleton.edu
\item Website: \url{http://tomfaulkenberry.github.io}
\item Phone: 254-968-9816
\end{itemize}

\section*{Course description}
\label{sec-2}

Statistical methods are the primary tool for research in psychology.  
They are what allow us as researchers to make consistent, data-driven 
decisions.  As such, this is an extremely important course and one that I 
take very seriously as your professor.

The topics we will cover this semester will include descriptive statistics 
(how we describe data) and inferential statistics (how we make decisions 
about data).  Specifically, this includes central tendency, variability, 
correlation, the distinction between populations and samples, hypothesis 
testing, statistical significance, and a variety of inferential tests 
that we can apply to data, including t-tests and analysis of variance.

\section*{Course materials}
\label{sec-3}
\begin{itemize}
\item \emph{Statistics for the Behavioral Sciences (10th ed.)} by Gravetter and Wallnau \href{http://www.amazon.com/Statistics-Behavioral-Sciences-MindTap-Psychology/dp/1305504917/}{Amazon link}
\item Note:  older editions of this textbook are just fine.  Please feel free to find a used copy of an older edition on Amazon; it will save you a LOT of money!
\end{itemize}

\section*{Student learning outcomes}
\label{sec-4}
\begin{enumerate}
\item Identify variables under study (including independent and dependent variables)
\item Choose appropriate measures of descriptive statistics
\item Select and perform appropriate inferential statistics
\item Draw appropriate statistical conclusions from results of analyses
\end{enumerate}

\section*{Requirements and grading}
\label{sec-5}
\begin{itemize}
\item Exam 1 (100 pts)
\item Exam 2 (100 pts)
\item Exam 3 (100 pts)
\item Final exam (100 pts)
\item Unit quizzes (100 pts)
\item \emph{Total = 500 points}
\end{itemize}

Grades will be assigned based on the percentage of points you accumulate out of these 600 points.  I will use the standard grading scale of A=90\%, B=80\%, etc.

\subsection*{Exams (80\% of grade)}
\label{sec-5-1}
There will be four total exams throughout the semester, occurring approximately once every three to four weeks.  They will cover material from lectures, quizzes, and homework exercises.  Exam questions will be a mix of multiple choice and short answer.  Exams are due by 11:59 pm on their due date.  Each exam will have a time limit (2 hours) and may only be attempted once.

Due dates:

\begin{itemize}
\item Exam 1 (Wednesday, June 28 at 11:59 pm)
\item Exam 2 (Friday, July 14 at 11:59 pm)
\item Exam 3 (Friday, August 4 at 11:59 pm)
\item Final exam (Wednesday, August 9 at 11:59 pm)
\end{itemize}

\subsection*{Unit quizzes (20\% of grade)}
\label{sec-5-2}
At the end of each unit, you will complete a 20-question quiz over the content of that unit.  The quizzes will have a generous time limit (60 minutes), and may be attempted twice.  In the event that you complete two attempts, your last graded attempt is the one that will be recorded.  Each quiz counts for 10 possible points.  Since there are 10 units, you will earn up to 100 points for your quiz grade.

\section*{Course Communication}
\label{sec-6}

I am your primary resource for this course. I AM an experimental psychologist, so I do the stuff I teach about daily. Hence, my primary interest is for you to learn this material and do well in the course. If you have a question, always feel free to stop by my office and visit.  If you require electronic communication, please email me.  I will be sending periodic emails to each of you that update you on course progress, due dates, etc.  It is imperative that you check your Tarleton email regularly so that you don't miss any of these messages.

\section*{University Policy on "F" Grades}
\label{sec-7}
Beginning in Fall 2015, Tarleton will begin differentiating between a failed grade in a class because a student never attended (F0 grade), stopped attending at some point in the semester (FX grade), or because the student did not pass the course (F) but attended the entire semester. These grades will be noted on the official transcript. Stopping or never attending class can result in the student having to return aid monies received.  For more information see the Tarleton Financial Aid website.

\section*{Academic Honesty}
\label{sec-8}

Tarleton State University expects its students to maintain high standards of personal and scholarly conduct. Students guilty of academic dishonesty are subject to disciplinary action. Cheating, plagiarism (submitting another person’s materials or ideas as one’s own), or doing work for another person who will receive academic credit are all disallowed. This includes the use of unauthorized books, notebooks, or other sources in order to secure of give help during an examination, the unauthorized copying of examinations, assignments, reports, or term papers, or the presentation of unacknowledged material as if it were the student’s own work. Disciplinary action may be taken beyond the academic discipline administered by the faculty member who teaches the course in which the cheating took place.

In particular, any exam taken online must be completed without the aid of any unauthorized resource (including using any search engine, Google, etc.).  Authorized resources are limited only to the official textbook and any lecture notes from the course.  Any other authorized resources will be provided to you before the exam.  The minimum sanction for violation of this policy is a grade of 0 on the affected exam.

Each student’s honesty and integrity are taken for granted. However, if I find evidence of academic misconduct I will pursue the matter to the fullest extent permitted by the university. ACADEMIC MISCONDUCT OR DISHONESTY WILL RESULT IN A GRADE OF F FOR THE COURSE.  Students are strongly advised to avoid even the \emph{appearance} of academic misconduct. 

\section*{Academic Affairs Core Value Statements}
\label{sec-9}

\subsection*{Academic Integrity Statement}
\label{sec-9-1}
Tarleton State University's core values are integrity, leadership, tradition, civility, excellence, and service.  Central to these values is integrity, which is maintaining a high standard of personal and scholarly conduct.  Academic integrity represents the choice to uphold ethical responsibility for one’s learning within the academic community, regardless of audience or situation.

\subsection*{Academic Civility Statement}
\label{sec-9-2}
Students are expected to interact with professors and peers in a respectful manner that enhances the learning environment. Professors may require a student who deviates from this expectation to leave the face-to-face (or virtual) classroom learning environment for that particular class session (and potentially subsequent class sessions) for a specific amount of time. In addition, the professor might consider the university disciplinary process (for Academic Affairs/Student Life) for egregious or continued disruptive behavior.

\subsection*{Academic Excellence Statement}
\label{sec-9-3}
Tarleton holds high expectations for students to assume responsibility for their own individual learning. Students are also expected to achieve academic excellence by:
\begin{itemize}
\item honoring Tarleton’s core values, upholding high standards of habit and behavior.
\item maintaining excellence through class attendance and punctuality, preparing for active participation in all learning experiences.
\item putting forth their best individual effort.
\item continually improving as independent learners.
\item engaging in extracurricular opportunities that encourage personal and academic growth.
\item reflecting critically upon feedback and applying these lessons to meet future challenges.
\end{itemize}

\section*{Students with Disabilities Policy}
\label{sec-10}

It is the policy of Tarleton State University to comply with the Americans
with Disabilities Act and other applicable laws. If you are a student with a
disability seeking accommodations for this course, please contact Trina
Geye, Director of Student Disability Services, at 254.968.9400 or
geye@tarleton.edu. Student Disability Services is
located in Math 201. More information can be found at www.tarleton.edu/sds or in the University Catalog.


\textbf{\textbf{Note:  any changes to this syllabus will be communicated to you by the instructor!}}

\section*{Semester Schedule}
\label{sec-11}
\begin{center}
\begin{tabular}{rll}
Unit & Topic & Due date\\
\hline
1 & Displaying data & Friday, June 16\\
2 & Descriptives 1: central tendency, variation, and z-scores & Tuesday, June 20\\
3 & Descriptives 2: correlation & Friday, June 23\\
 & \textbf{Exam 1} & Wednesday, June 28\\
4 & The normal distribution: measuring likelihood & Monday, July 3\\
5 & The logic of hypothesis testing & Thursday, July 6\\
6 & Testing means of samples of \textbf{known} populations: $z$-tests & Tuesday, July 11\\
 & \textbf{Exam 2} & Friday, July 14\\
7 & Testing means of samples of \textbf{unknown} populations: $t$-tests & Wednesday, July 19\\
8 & More $t$-tests (independent samples, etc.) & Monday, July 24\\
9 & Analysis of variance (ANOVA): one independent variable & Thursday, July 27\\
10 & Analysis of variance (ANOVA): two independent variables & Tuesday, July 31\\
 & \textbf{Exam 3} & Friday, August 4\\
 & \textbf{Final exam} & Wednesday, August 9\\
 &  & \\
\end{tabular}
\end{center}
% Emacs 25.2.1 (Org mode 8.2.10)
\end{document}