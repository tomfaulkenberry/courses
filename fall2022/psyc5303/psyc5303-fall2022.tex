% Created 2022-08-22 Mon 09:56
% Intended LaTeX compiler: pdflatex
\documentclass[10pt]{article}
\usepackage[utf8]{inputenc}
\usepackage[T1]{fontenc}
\usepackage{graphicx}
\usepackage{longtable}
\usepackage{wrapfig}
\usepackage{rotating}
\usepackage[normalem]{ulem}
\usepackage{amsmath}
\usepackage{amssymb}
\usepackage{capt-of}
\usepackage{hyperref}
\usepackage[left=1in,right=1in,bottom=1in,top=1in]{geometry}
\setlength{\parindent}{0pt}
\setlength{\parskip}{2mm}
\date{Fall 2022}
\title{PSYC 5303: Theories of Learning}
\hypersetup{
 pdfauthor={},
 pdftitle={PSYC 5303: Theories of Learning},
 pdfkeywords={},
 pdfsubject={},
 pdfcreator={Emacs 28.1 (Org mode 9.5.2)}, 
 pdflang={English}}
\begin{document}

\maketitle

\section*{Contact info}
\label{sec:org09a7eff}
\begin{itemize}
\item Professor: Thomas J. Faulkenberry, Ph.D
\item Office: Math 301
\item Office hours: \emph{TBA}
\item Email: faulkenberry@tarleton.edu
\item Website: \url{http://tomfaulkenberry.github.io}
\item Phone: 254-968-9816
\end{itemize}

\section*{Course description}
\label{sec:org4b1f27b}

In this introductory level graduate course, we will survey a wide range of persepectives (both historical and current) on the theoretical principles of learning, memory, and cognition.  To accomplish this, you will read a variety of book chapters and journal articles that are chosen to help you develop a deep knowledge base of the theories, empirical methods, and applications of human learning, memory, and cognition. Additionally, students will get practice in writing and presenting scientific information.

\section*{Course materials}
\label{sec:org7b49dda}

There are several relevant textbooks for this course, but some are hard to find, expensive, or both!  Hence, I will be providing copies of selected chapters as we need them.  However, you are welcome to find your own copies of these books, as they are very good.

\begin{itemize}
\item Bower, G. H., \& Hilgard, E. R. (1981). \textbf{Theories of Learning} (5th ed.). Englewood Cliffs, NJ: Prentice-Hall, Inc. (\href{https://www.amazon.com/Theories-Learning-5th-Gordon-Bower/dp/0139144323}{Amazon link})
\item Eichenbaum, H. (2008). \textbf{Learning \& Memory}. New York: W. W. Norton Company. (\href{https://www.amazon.com/Learning-Memory-Howard-Eichenbaum/dp/0393924475}{Amazon link})
\item Lieberman, D. A. (2021). \textbf{Learning and memory} (2nd ed.). Cambridge University press (\href{https://www.amazon.com/Learning-Memory-David-Lieberman/dp/1108428614/}{Amazon link})
\item Neath, I., \& Surprenant, A. M. (2003). \textbf{Human memory} (2nd ed.). Belmont, CA: Wadsworth. (\href{https://www.amazon.com/Human-Memory-second-Neath-Surprenant/dp/B00BUWB592/}{Amazon link})
\end{itemize}

\section*{Student learning outcomes}
\label{sec:orga6eedaf}

\begin{enumerate}
\item Discuss the experimental procedures used to study human learning, memory, and cognition.
\item Demonstrate an understanding of both contemporary and historically important conceptual frameworks in the study of human learning, memory, and cognition.
\item Discuss the theoretical arguments that have motivated research on human learning, memory, and cognition.
\item Write and present summaries of relevant modern research papers in human learning, memory, and cognition.
\end{enumerate}

\section*{Structure of class meetings and grading}
\label{sec:org29725e2}

This is a graduate-level course. Regular attendance and full participation in each weekly class session are essential requirements. You must carefully, thoroughly, and thoughtfully complete all reading and writing assignments \textbf{prior} to each class.  
Each week will be devoted to a single topic, supported by both secondary source material (i.e., textbook chapters) and primary source material (i.e., empirical journal articles). Before attending class any given week, you will need to complete \emph{two readings} -- one book chapter, and one empirical article.  All reading materials will be provided on Canvas.

I will assess your understanding of these readings two ways.
\begin{itemize}
\item First, at the beginning of class you will complete a short in-class quiz over the textbook reading. Students attending in-person in Stephenville will complete the quiz on paper, whereas those attending via Zoom will complete the quiz on Canvas.
\item Second, you will turn in a one-page article summary about \textbf{one} of the week's two assigned journal articles.  Your summary must contain the following:
\begin{itemize}
\item describe the motivation for the experiment (i.e., what problem was the paper attempting to solve?)
\item a brief description of the method, particularly describing the key dependent variables (what was measured) and independent variables (what was manipulated)
\item a conceptual description of the results (don't list the statistical results, just tell me the "big picture")
\end{itemize}
\end{itemize}

After completing the in-class quiz and submitting your journal article summary, we will have a traditional lecture/discussion over the week's textbook reading.

After this lecture, we will take a short break, then conclude each class session with two student-led discussions over the assigned journal article readings. Here, two students will serve as discussion leaders.  This means they will have carefully read their assigned paper, read additional material as necessary, and prepared an oral presentation (with appropriate slides and/or handouts).  Class presentations should be short and to the point (no more than 10 minutes for the presentation and an additional 10 minutes for discussion, giving a total of 20 minutes for each presenter).

In addition to the weekly contributions during class, there will be two in-class exams (one during our Week 8 meeting and one during our Week 15 meeting).  The week prior to each exam you will be given a set of 10-15 potential essay questions. Students may work together when preparing for the exam, but the exam will be completed individually. Answers should demonstrate "graduate-level thinking" -- this means that when you answer a question, you will support your arguments with empirical evidence (not just anecdotes or statements of what \emph{you believe}) and give descriptions of specific experiments that lead to your conclusions. Six questions from the original list will be selected for the exam, and you will be asked to answer 5 of them. The questions will be based on material covered in lecture, assigned readings, and class discussions. The exams will be given in class, and you will not be able to consult any external materials.  In other words, you will have to retrieve from your memory your responses to the practice exam.  Violations will result in a grade of ZERO on the exam.

Note -- students attending via Zoom will use the Respondus lock-down browser to complete the exam during class.

These various components of your class performance will contribute to your grade as follows:

\begin{itemize}
\item Midterm exam (25\%)
\item Final exam (25\%)
\item Weekly quizzes (20\%)
\item Weekly article summaries (20\%)
\item Leading discussions of assigned readings (10\%)
\end{itemize}

\section*{Additional notes for Zoom attendees}
\label{sec:orgb75deab}

This class allows Zoom attendance for students outside the Stephenville campus area. Students enrolled in the Stephenville section (010) should plan to attend \emph{in person}. However, students in the Zoom section (011) may attend either in person or by Zoom. The Zoom link is posted prominently on the Canvas page for the course. Attending by Zoom requires that you have access to a computer with a camera, microphone, and speakers/headphones. Please set up a quiet workspace so that you can concentrate fully on the class session. Your camera should be turned ON during the entire class session (except during breaks). All quizzes and exams will be completed on Canvas -- they will be opened at the exact moment that the corresponding "paper" version is handed out in the Stephenville class. Note that the two exams (Midterm and Final) will require the use of the Respondus lock-down browser (i.e., the online equivalent of a "closed book exam").

\section*{Course Communication}
\label{sec:orgc4c4c01}

This course is designed to be an intensive, interactive course on modern statistical methods and experimental design.  That means that I will be available for one-on-one consultation most any time.  Just stop by my office or give me a call.

All official course communication (questions, setting up a meeting, etc.) will be conducted by email.  Any time you need to contact me, feel free to send me an email at faulkenberry@tarleton.edu.  I only ask that you adhere to two guidelines:
\begin{itemize}
\item please include the course number (PSYC 5303) in the subject line.  For example, one good way to do this is:  Subject: [PSYC 5303] Question about week 3 assignment.
\item please use proper email etiquette.  Include a salutation (e.g., Dear Dr. Faulkenberry), complete sentences, and a closing (e.g., "Regards, Your Name").  You might be surprised how many times I get an email from a nondescript email address with no indication from WHOM the email was sent!
\end{itemize}

Also, I will be sending periodic emails to each of you that update you on course progress, due dates, etc.  It is imperative that you check your \emph{Tarleton email address} regularly so that you don't miss any of these messages.

\section*{University Policy on "F" Grades}
\label{sec:org4172589}
Beginning in Fall 2015, Tarleton will begin differentiating between a failed grade in a class because a student never attended (F0 grade), stopped attending at some point in the semester (FX grade), or because the student did not pass the course (F) but attended the entire semester. These grades will be noted on the official transcript. Stopping or never attending class can result in the student having to return aid monies received.  For more information see the Tarleton Financial Aid website.

\section*{Academic Honesty}
\label{sec:org6a91636}

Tarleton State University expects its students to maintain high standards of personal and scholarly conduct. Students guilty of academic dishonesty are subject to disciplinary action. Cheating, plagiarism (submitting another person’s materials or ideas as one’s own), or doing work for another person who will receive academic credit are all disallowed. This includes the use of unauthorized books, notebooks, or other sources in order to secure of give help during an examination, the unauthorized copying of examinations, assignments, reports, or term papers, or the presentation of unacknowledged material as if it were the student’s own work. Disciplinary action may be taken beyond the academic discipline administered by the faculty member who teaches the course in which the cheating took place.

In particular, any exam taken online must be completed without the aid of any unauthorized resource (including using any search engine, Google, etc.).  Authorized resources are limited only to the official textbook and any lecture notes from the course.  Any other authorized resources will be provided to you before the exam.  The minimum sanction for violation of this policy is a grade of 0 on the affected exam.

Each student’s honesty and integrity are taken for granted. However, if I find evidence of academic misconduct I will pursue the matter to the fullest extent permitted by the university. ACADEMIC MISCONDUCT OR DISHONESTY WILL RESULT IN A GRADE OF F FOR THE COURSE.  Students are strongly advised to avoid even the \emph{appearance} of academic misconduct. 

\section*{Academic Affairs Core Value Statements}
\label{sec:org12e49e6}

\subsection*{Academic Integrity Statement}
\label{sec:org539da16}
Tarleton State University's core values are integrity, leadership, tradition, civility, excellence, and service.  Central to these values is integrity, which is maintaining a high standard of personal and scholarly conduct.  Academic integrity represents the choice to uphold ethical responsibility for one’s learning within the academic community, regardless of audience or situation.

\subsection*{Academic Civility Statement}
\label{sec:org79818c4}
Students are expected to interact with professors and peers in a respectful manner that enhances the learning environment. Professors may require a student who deviates from this expectation to leave the face-to-face (or virtual) classroom learning environment for that particular class session (and potentially subsequent class sessions) for a specific amount of time. In addition, the professor might consider the university disciplinary process (for Academic Affairs/Student Life) for egregious or continued disruptive behavior.

\subsection*{Academic Excellence Statement}
\label{sec:org992decd}
Tarleton holds high expectations for students to assume responsibility for their own individual learning. Students are also expected to achieve academic excellence by:
\begin{itemize}
\item honoring Tarleton’s core values, upholding high standards of habit and behavior.
\item maintaining excellence through class attendance and punctuality, preparing for active participation in all learning experiences.
\item putting forth their best individual effort.
\item continually improving as independent learners.
\item engaging in extracurricular opportunities that encourage personal and academic growth.
\item reflecting critically upon feedback and applying these lessons to meet future challenges.
\end{itemize}

\section*{Students with Disabilities Policy}
\label{sec:org73cb26e}

It is the policy of Tarleton State University to comply with the Americans with Disabilities  Act (www.ada.gov) and other applicable laws.  If you are a student with a disability seeking accommodations for this course, please contact the Center for Access and Academic Testing, at 254.968.9400 or caat@tarleton.edu. The office is located in Math 201. More information can be found at www.tarleton.edu/caat or in the University Catalog.​

\textbf{Note:  any changes to this syllabus will be communicated to you by the instructor!}

\section*{Schedule at a glance}
\label{sec:org7ed4012}

\begin{center}
\begin{tabular}{rll}
Week & Date & Topic(s) covered\\
\hline
1 & 8/23 & Introduction to course / what is learning?\\
2 & 8/30 & Mechanisms of classical conditioning\\
3 & 9/6 & Mechanisms of instrumental conditioning\\
4 & 9/13 & Mathematical models of conditioning\\
5 & 9/20 & Human associative learning\\
6 & 9/27 & Classical models of human memory\\
7 & 10/4 & Working memory\\
8 & 10/11 & \emph{Midterm exam}\\
9 & 10/18 & Encoding and retrieval processes\\
10 & 10/25 & Forgetting\\
11 & 11/1 & Implicit memory\\
12 & 11/8 & Recognition\\
13 & 11/15 & Knowledge structures in long term memory\\
14 & 11/22 & \emph{No class for Thanksgiving holiday}\\
15 & 11/29 & Topic TBA\\
16 & 12/7 & \emph{Final exam}\\
\end{tabular}
\end{center}
\end{document}